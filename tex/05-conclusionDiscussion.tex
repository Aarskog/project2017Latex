\chapter{Conclusion}
In this project several aspects of robotics have been considered. The first topic is visualization and simulation where a virtual model of a given robot is made. This includes measurements, mass and moments of inertia. Some of the measurements and mass is only guessed or estimated and it may not result in a perfect recreation of the actual robot. The moments of inertia are also found based on simplifications and if one want a more realistic behaviour of the robot it may be by recalculating this. The simulation environment Gazebo is a good software with much support online in forums and wikis. There was however some trouble when implementing the moments of inertia that would lead to the robot not spawning properly, because of too low inertia of the two fingers.\\\\
The second part that was presented was the kinematics problem where the forward kinematics was found. The DH convention was used to find this. Then the inverse kinematics problem were solved by using the MATLAB robotics toolbox\cite{MatlabRobTool}. The toolbox gave good and precise solutions and was easy to use. One thing that may not be so good is that it uses about 100ms per set point which may be too slow for an online planner\cite{spong}, but if offline planning is the purpose the time consumption does not matter.
\\\\
The next part is the control and motion planning. The PD controller with gravity compensation is a basic controller which worked well for this case, but a better controller should be considered. Both \cite{spong} and \cite{Siciliano} are good resources when considering more advanced controllers. This project includes a way to plan a simple path with a single obstacle present in it. The trajectory is not found though, because of the obstacle one does not want only the end effector to avoid the obstacle, but the rest of the robot manipulator must avoid it as well which is a more complex problem and should be included in further work. In fact, the motion planning is a very complex task and \cite{spong} states that path and trajectory planning is one of the most difficult problems in computer science.